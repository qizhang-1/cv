\documentclass[11pt]{article}
\usepackage[letterpaper,margin=0.3in]{geometry}
\usepackage[
pdftitle = ``Resume of QI ZHANG'',
pdfauthor = ``QI ZHANG''
]{hyperref}
\usepackage{amsmath,amssymb,amsfonts,array,tabularx,enumitem,textcomp,titlesec,parskip,times,txfonts,pifont}
\usepackage{titlesec} 
\usepackage{fancyhdr}
\pagestyle{empty}
\setlength{\parskip}{0pt}
\setlength{\topskip}{0pt}
\setlength{\floatsep}{0pt}
\setlength{\textfloatsep}{0pt}
\setlength{\intextsep}{0pt}
\setlist{nolistsep}
\newcommand{\sep}{\ding{118}\;}
\newcommand{\QZ}{\textbf{Q. Zhang}}
\newcommand{\DJB}{D. Bodony}

\begin{document}

\begin{center}
{\Huge\textbf{\textsc{{Qi Zhang}}}} \\[5pt]
1600 Wickersham Ln. Apt 2089, Austin, TX 78741 $\text{ }\bullet\text{ }$ (716) 238-1959 $\text{ }\bullet\text{ }$  bl1008@gmail.com $\text{ }\bullet\text{ }$ U.S. Permanent Resident 
\end{center}
 \rule[0.8em]{\textwidth}{0.5pt} \\[-11pt]
 \rule[0.8em]{\textwidth}{0.5pt} \\[0pt]
%--- PROFESSIONAL PROFILE
\textbf{\textsc{\LARGE{Professional Profile Summary}}} \\ \rule[0.8em]{\textwidth}{0.5pt} \\[-20pt]
\begin{itemize}[topsep=0pt,partopsep=0pt,leftmargin=16pt]
\item Expertise in of fluid dynamics, aerodynamics, aeroacoustics, control theory and differential equations 
\item Excellent in different algorithms, parallel programming and objective and oriented programming
\item Proficiency in Java, C/C++, Fortran and Python programming languages
\item Excellent in team work, analytical thinking and difficult problem solving skills
%\item Extremely high motivation in self-learning
\end{itemize}
\vspace{4pt}
\textbf{\textsc{\LARGE{Highlighted Skills}}} \\ \rule[0.8em]{\textwidth}{0.5pt} \\[-18pt]
\vspace{-4pt}
\begin{itemize}[topsep=0pt,partopsep=0pt,leftmargin=16pt]
\item
C/C++, 
Fortran, 
Java,
Python, 
Matlab, 
Unix Shell Scripting, 
\LaTeX,
SDLC tools: git and SVN
\end{itemize}
\vspace{4pt}
%--- Employment
\textbf{\textsc{\LARGE{Employment}}} \\ \rule[0.8em]{\textwidth}{0.5pt} \\[-16pt]
\\
\begin{tabularx}{\textwidth}{>{\raggedright}X>{\raggedleft}p{0.25\textwidth}}
\textbf{Institute of Computational Engineering Sciences, University of Texas at Austin } \\
{\em Research Associate}\\
&
Austin, TX \\
March 2016 -- Present
\tabularnewline[-12pt]
\multicolumn{2}{p{0.98\textwidth}}{
\textbf{Project: Machine learning approach in turbulence wall-pressure modeling}
\begin{itemize}[leftmargin = 16pt]
\item Lead a 1\,M USD NASA \& Sandia National Lab funded project
\item Implemented DES and LES/DES hybrid turbulent wall models and designed unit tests (C/C++) in SU2 code
\item Wrote various post-processing tools (Python \& Java) for data parsing and statistical analysis
\item Speeded up high-fidelity simulation data generation 100+ times for machine learning studies at the Sandia National Lab  
\end{itemize}
}
\\
\vspace{-20pt}
\textbf{Department of Aerospace Engineering, University of Illinois at Urbana-Champaign } \\
{\em Research Associate} \\
&
Urbana, IL \\
March 2014 -- March 2016 
\tabularnewline[-12pt]
\multicolumn{2}{p{0.98\textwidth}}{
\textbf{Project: Exascale Simulation of Plasma-Coupled Combustion (20\,M USD funding)}
\begin{itemize}[leftmargin = 16pt]
\item Developed a parallel 2-D/3-D compressible Navier-Stokes equation solver, 5,000+ lines with multiple features
\item Improved large-scale data post processing 5+ times faster using novel algorithms in filtering non-physical data 
\item Jet-in-cross flow simulation results highlighted on the 2014 DOE PSSAP2 annual report 
\end{itemize}
\textbf{Project: Performance Analysis and Optimization of a High-Order MPI CFD Application}
\begin{itemize}[leftmargin = 16pt]
\item Analyzed the performance of the in-house MPI large-scale CFD solver using TAU and PAPI (hardware counters)
\item Identified performance bottlenecks to be large quantity of memory accesses and the lack of vectorization
\item Optimized the CFD codes and obtained a 50\%+ reduction in the number of memory loads Intel Xeon E5-2680 processor
\end{itemize}
}
\end{tabularx}

%--- EDUCATION
\vspace{-10pt}
\textbf{\textsc{\LARGE{Education}}} \\ \rule[0.8em]{\textwidth}{0.5pt} \\[-16pt]

\begin{tabularx}{\textwidth}{>{\raggedright}X>{\raggedleft}p{0.16\textwidth}}
\textbf{University of Illinois at Urbana-Champaign, Urbana, IL} \\
\textbf{Ph.D.} in Aerospace Engineering $\quad\qquad$ Research Topic: \textit{Computational Science and Engineering}\\%Minor in Computational Science and Engineering 
Thesis: \emph{Direct Numerical Simulation and Analytical Modeling of 3-D Honeycomb Acoustic Liners.} 
&
May 2014\\
\textbf{GPA 4.00/4.00}
\tabularnewline[4pt]
%\textbf{University of Illinois at Urbana-Champaign, Urbana, IL} \\
%\textbf{M.S.} in Aerospace Engineering 
%%\emph{Thesis: Numerical Simulation of 2-D Acoustic Liners with High Speed Grazing Flow. }
%&
%December 2009\\
%\textbf{GPA 3.98/4.00}
%\tabularnewline[4pt]
\textbf{Fudan University, Shanghai, China} \\
\textbf{B.S.} in Theoretical and Applied Mechanics %\\
% \textbf{Standardized Test Scores:} GRE -- Verbal: 650/800, Quantitative: 800/800. TOEFL -- 112/120.
&
July 2007 \\
\textbf{GPA 3.89/4.00}
\end{tabularx}

\vspace{6pt}

%--- EXPERIENCE
\textbf{\textsc{\LARGE{Professional Experience}}} \\ \rule[0.8em]{\textwidth}{0.5pt} \\[-16pt]

\begin{tabularx}{\textwidth}{>{\raggedright}X>{\raggedleft}p{0.25\textwidth}}
\textbf{Department of Aerospace Engineering, University of Illinois at Urbana-Champaign } \\
{{\em Research Assistant}} \\
&
August 2008 -- March 2014
\tabularnewline[-12pt]
\multicolumn{2}{p{0.98\textwidth}}{
\textbf{Project: Liner eduction methodology using large-eddy simulation}
\begin{itemize}[leftmargin = 16pt]
%\item Developed and implemented SBP-SAT finite difference numerical algorithm for compressible Navier-Stokes equations
%\item Numerical simulation results of a jet-in-crossflow for the DOE PSAAP2 Center featured on the 2014 SSAP Annual Report
%\item Speeded up large-scale data post processing 7+ times using novel algorithms in filtering non-physical data 
%\item Developed an optimized reduced order model with multiple applications to the fluid dynamics and acoustics problems
%\item Investigated the flow properties in Grazing Flow Impedance Tube (GFIT)  with a 4.3\% error between the experimental measurement.   (Data obtained from NASA Langley; grid generated via Gridgen$\text superscript{\textregistered}$; simulation performed via Fluent$\textsuperscript{\textregistered}$)
%\item Performed the world \textbf{first} large-scale DNS of sound induced flow through orifice under high speed turbulent boundary layer using scalable MPI codes on parallel environments to investigate active noise reduction 
%\item Discovered several new flow features of the interactions between the turbulent boundary layer and acoustic field inside/near the circular orifice
%\item Validated the numerical predictions with the experimental results via multiple physics quantity comparisons (i.e.  impedance, energy spectra)
%\item Extended the numerical predictions beyond the experimental capability with proper certainty and error quantifications 
%\item Analyzed the additional drag induced by the acoustic liners under different sound pressure levels 
%\item Developed an 1-D analytical model for NASA honeycomb liner impedance prediction with turbulent boundary layer %$\qquad$ (Ph.D Dissertation Focus)
%\item Conducted a Mach 1.3 NASA supersonic nozzle jet simulation and predicted the sound (noise) radiation
\item Developed, tested and debugged a high-order finite volume numerical toolkit, 3,000+ lines
%\item Performed high-speed jet engine and aerodynamics flow simulations in different Linux supercomputing platforms
%\item Developed and implemented SBP-SAT finite difference numerical algorithm for compressible Navier-Stokes equations
\item Developed an optimized reduced order model with multiple applications to the fluid dynamics and acoustics problems
\item Analyzed the simulation data in both time and frequency domain via Fast Fourier Transform (FFT)
\item Designed and performed the world first large-scale DNS of sound induced under high-speed turbulent flow
\item Designed a highly-optimized ODE driven reduced-order models for acoustic liner in noise control
\end{itemize}
}
%\tabularnewline[-10pt]
%\textbf{{\em Teaching Assistant}} (Computational Aeroacoustics) \\
%&
%January 2013 -- May 2013
%\tabularnewline[-20pt]
%\multicolumn{2}{p{0.98\textwidth}}{
%\begin{itemize}[leftmargin = 16pt]
%\item Developed a 1-D wave equation solver 
%\item Developed a parallel 2-D/3-D compressible Navier-Stokes equation solver
%\end{itemize}
%}
\tabularnewline[-10pt]
{{\em Teaching Assistant and Class Tutor}} (Data Structure and Algorithms) \\
&
September 2011 -- May 2014
\tabularnewline[-20pt]
\multicolumn{2}{p{0.98\textwidth}}{
\begin{itemize}[leftmargin = 16pt]
\item Instructed and provided guidance for the students to their class projects  (Java \& C/C++)
\end{itemize}
}
\end{tabularx}



\vspace{-10pt}
%%--- PUBLICATIONS & CONFERENCES
%\textbf{\textsc{\LARGE{Publications}}} \\ \rule[0.8em]{\textwidth}{0.5pt} \\[0pt]
%\vspace{-30pt}
%\begin{itemize}[topsep=6pt,partopsep=0pt,leftmargin=16pt]
%\item 5 top journal articles and 15 conference proceedings (150+ citations) 
%%\item 100+  peer-review manuscript experiences in top ranked journals and conference proceedings
%\end{itemize}
%%\begin{tabularx}{\textwidth}{>{\raggedright}X>{\raggedleft}p{0.25\textwidth}}



%
%%\item \textbf{Softwares}: 
%%\LaTeX, 
%%Mathematica,
%%ANSYS
%%Fluent/Gambit, 
%%Gridgen/Pointwise, 
%%Eclipse, 
%%AutoCAD,
%%Tecplot
%%\item \textbf{Courses}: 
%%Numerical Methods for PDEs, 
%%Computational Fluid Mechanics, 
%%Turbulence, 
%%Inviscid \& Viscous Flow, 
%%Numerical Methods,
%%Heat Transfer,
%%Data Structures,
%%Computational Aeroacoustics,
%%Finite/Boundary Element Methods,
%%Probability Theory
%%Statistics
%%Real Analysis
%%Complex Analysis
%%Partial Differential Equations

\vspace{6pt}

%--- HONORS & AWARDS
\textbf{\textsc{\LARGE{Publications, Honors \& Awards}}} \\ \rule[0.8em]{\textwidth}{0.5pt} \\[-18pt]

\begin{itemize}[topsep=0pt,partopsep=0pt,leftmargin=16pt]
\item 5 top journal articles and 15 conference proceedings (150+ citations) 
\item 2016 Institute of Computational Engineering Sciences Postdoc Fellowship (ranked 1st)
%\item 2009 AIAA student conference Region III (Midwest) Winner 
%\item 2004 -- 2007 People's Fellowship from Fudan University First Prize (top 3\%)
\end{itemize}

\vspace{2pt}
%%--- LEADERSHIP
%\textbf{\textsc{\LARGE{Visa Status}}} \\ \rule[0.8em]{\textwidth}{0.5pt} \\[-18pt]
%\begin{itemize}[topsep=0pt,partopsep=0pt,leftmargin=16pt]
%\item U.S. Permanent Resident
%%\item 2004 -- 2006 People's Fellowship 2nd Prize
%\end{itemize}
%

%--- ACTIVITIES
%\textbf{\textsc{\LARGE{Memberships}}} \\ \rule[0.8em]{\textwidth}{0.5pt} \\[-20pt]

%American Physical Society \sep
%American Society of Mechanical Engineers \sep
%American Institute of Aeronautics and Astronautics \sep
%Linux Users' Group 
%National Service Scheme (India), 2007--2008 \sep
%Physics Society Chennai \sep
%Web Design \sep
%Open Source Programming\sep
%Master Swimming Club

\end{document}


% \item R.~Vishnampet, J.~B.~Freund \& D.~Bodony. Navier-Stokes adjoint accuracy for aeroacoustic flow control and analysis. {\em APS
%     Meeting Abstracts}, \textbf{57}, 2012.
% \item R.~Vishnampet \& D.~Saintillan. Concentration instability of sedimenting spheres in a second-order fluid. {\em Phys. Fluids}, \textbf{24}, 2012.
% \item A.~Narasimhan \& R.~Vishnampet. Effect of choroidal blood flow on transscleral retinal drug delivery using a porous medium
%   model. {\em Int. J. Heat Mass Transfer}, \textbf{55}:5665--5672, 2012.
% \item R.~Vishnampet, A.~Narasimhan, \& V.~Babu. High Rayleigh Number Natural Convection Inside 2D Porous Enclosures Using the
%   Lattice Boltzmann Method. {\em J. Heat Transfer}, \textbf{133}:062501, 2011.
% \item V.~Deepesh, R.~J.~Pardikar, A.~Sricharan, V.~G.~Ramanathan, S.~Chakravarthy, \& K.~Balasubramaniam. Automatic Defect Recognition System for Real Time
% Radioscopy of Hancock Valve Welds. {\em Journal of Nondestructive Testing \& Evaluation}, \textbf{9}:1--6, 2010.

% \end{enumerate}
