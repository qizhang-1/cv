\documentclass[11pt]{article}
\usepackage[letterpaper,margin=0.64in]{geometry}
\usepackage[
pdftitle = ``Resume of QI ZHANG'',
pdfauthor = ``Charlie  ZHANG''
]{hyperref}
\usepackage{amsmath,amssymb,amsfonts,array,tabularx,enumitem,textcomp,titlesec,parskip,times,txfonts,pifont}
\usepackage{titlesec} 
\usepackage{fancyhdr}
\pagestyle{empty}
\setlength{\parskip}{0pt}
\setlength{\topskip}{0pt}
\setlength{\floatsep}{0pt}
\setlength{\textfloatsep}{0pt}
\setlength{\intextsep}{0pt}
\setlist{nolistsep}
\newcommand{\sep}{\ding{118}\;}
\newcommand{\QZ}{\textbf{Q. Zhang}}
\newcommand{\DJB}{D. Bodony}

\begin{document}

\begin{center}
{\huge\textbf{\textsc{{Qi Zhang}}}} \\[5pt]
\textbf{Cell}: (716) 238-1959 $\text{ }\quad\bullet\bullet\bullet\bullet\bullet\quad\text{ }$ \textbf{E-mail}: \texttt{qi@ices.utexas.edu} \quad$\text{ }\bullet\bullet\bullet\bullet\bullet\quad\text{  }$ \textbf{Visa status}: Green Card 
\end{center}
 \rule[0.8em]{\textwidth}{0.5pt} \\[-11pt]
 \rule[0.8em]{\textwidth}{0.5pt} \\[0pt]
%--- PROFESSIONAL PROFILE
\textbf{\textsc{\LARGE{Profile Summary}}} \\ \rule[0.8em]{\textwidth}{0.5pt} \\[-20pt]
\begin{itemize}[topsep=0pt,partopsep=0pt,leftmargin=16pt]
\item Software engineer bringing 2+ years experience in academia with multiple programming languages.
\item Knowledge of the different machine learning algorithms with various applications. 
\item Supportive and enthusiastic team player dedicated to efficient difficult problem solving.
\item Willing to take ownership of core components.
\end{itemize}
\vspace{12pt}
\textbf{\textsc{\LARGE{Highlighted Skills}}} \\ \rule[0.8em]{\textwidth}{0.5pt} \\[-12pt]
\begin{tabularx}{\textwidth}{>{\raggedright}X>{\raggedleft}p{0.5\textwidth}}
\begin{itemize}[topsep=0pt,partopsep=0pt,leftmargin=16pt]
\item Parallel, Object Oriented Programming 
\item Java, C/C++, Python, Fortran, Scala, MySQL
\item Hadoop, Spark 
\end{itemize}
&
\begin{itemize}[topsep=0pt,partopsep=0pt,leftmargin=16pt]
\item Various machine learning algorithms
\item Machine learning APIs (i.e. sklearn, TensorFlow)
\item SDLC tools: git and SVN
\end{itemize}
\tabularnewline[-12pt]
\end{tabularx}
\vspace{6pt}


%--- EXPERIENCE
\textbf{\textsc{\LARGE{Professional Experience}}} \\ \rule[0.8em]{\textwidth}{0.5pt} \\[-16pt]
%--- Employment
%\textbf{\textsc{\LARGE{Employment}}} \\ \rule[0.8em]{\textwidth}{0.5pt} \\[-16pt]
\\
\begin{tabularx}{\textwidth}{>{\raggedright}X>{\raggedleft}p{0.27\textwidth}}
\textbf{Institute of Computational Engineering Sciences, University of Texas} \\
{\em Software Engineer and Research Associate}\\
&
Austin, TX \\
March 2016 -- September 2017
\tabularnewline[-12pt]
\multicolumn{2}{p{0.98\textwidth}}{
\textbf{Project: Machine learning feature selection in turbulence wall-pressure modeling}
\begin{itemize}[leftmargin = 16pt]
\item Designed and implemented multiple features for SU2 code (CFD code) on parallel clusters
\item Designed and implemented different unit tests (C/C++)
\item Implemented various post-processing tools (Python \& Java) for data parsing and statistical analysis
\item Speeded up data generation 100+ times for machine learning studies at the Sandia National Lab  
\end{itemize}
}
\\
\vspace{-12pt}
\textbf{PSAAP2 Center, University of Illinois } \\
{\em Software Engineer and Research Associate} \\
&
Urbana, IL \\
May 2014 -- March 2016 
\tabularnewline[-12pt]
\multicolumn{2}{p{0.98\textwidth}}{
\textbf{Project: Performance Analysis and Optimization of a High-Order MPI CFD Application}
\begin{itemize}[leftmargin = 16pt]
\item Analyzed the performance of the MPI large-scale parallel CFD solver using TAU and PAPI 
\item Identified performance bottlenecks to be large quantity of memory accesses and the lack of vectorization
\item Optimized the CFD codes and obtained a 50\%+ reduction in the number of memory loads 
\end{itemize}
\textbf{Project: Exascale Simulation of Plasma-Coupled Combustion}
\begin{itemize}[leftmargin = 16pt]
\item Developed a parallel 2-D/3-D compressible Navier-Stokes equation solver, 5,000+ lines with multiple features
\item Improved large-scale data post processing 5+ times faster using novel algorithms in filtering non-physical data 
\item Jet-in-cross flow simulation results highlighted on the 2014 DOE PSAAP2 annual report 
\end{itemize}
\textbf{Project: Actuator type and placement for jet noise reduction}
\begin{itemize}[leftmargin = 16pt]
\item Developed, tested and debugged a high-order finite volume numerical toolkit, 3,000+ lines
\item Performed high-speed jet engine and aerodynamics flow simulations in different supercomputing platforms
\item Analyzed the simulation data in both time and frequency domain via Fast Fourier Transform (FFT)
\end{itemize}
}
\end{tabularx}

%\\
%\vspace{-10pt}
%\textbf{Department of Aerospace Engineering, University of Illinois at Urbana-Champaign } \\
%{\em Research Assistant} \\
%&
%Urbana, IL \\
%August 2008 -- March 2014
%\tabularnewline[-12pt]
%\multicolumn{2}{p{0.98\textwidth}}{
%\textbf{Project: Liner eduction methodology using large-eddy simulation}
%\begin{itemize}[leftmargin = 16pt]
%\item Analyzed the simulation data in both time and frequency domain via Fast Fourier Transform (FFT)
%\item Designed a highly-optimized ODE driven reduced-order models for acoustic liner in noise control
%\end{itemize}
%}
%\tabularnewline[-10pt]
%{{\em Teaching Assistant and Class Tutor}} (Data Structure and Algorithms) \\
%&
%September 2011 -- May 2014
%\tabularnewline[-20pt]
%\multicolumn{2}{p{0.98\textwidth}}{
%\begin{itemize}[leftmargin = 16pt]
%\item Instructed and provided guidance for the students to their class projects  (Java \& C/C++)
%\end{itemize}
%}
\textbf{\textsc{\LARGE{Education}}} \\ \rule[0.8em]{\textwidth}{0.5pt} \\[-16pt]\\
\begin{tabularx}{\textwidth}{>{\raggedright}X>{\raggedleft}p{0.16\textwidth}}
\textbf{University of Illinois at Urbana-Champaign, Urbana, IL} \\
\textbf{Ph.D.} in Aerospace Engineering (Focus: {Computational Science and Engineering})\\%Minor in Computational Science and Engineering 
&
\textbf{GPA 4.00/4.00}
\tabularnewline[-8pt]
%\textbf{University of Illinois at Urbana-Champaign, Urbana, IL} \\
%\textbf{M.S.} in Aerospace Engineering 
%%\emph{Thesis: Numerical Simulation of 2-D Acoustic Liners with High Speed Grazing Flow. }
%&
%December 2009\\
%\textbf{GPA 3.98/4.00}
%\tabularnewline[4pt]
\textbf{Fudan University, Shanghai, China} \\
\textbf{B.S.} in Theoretical and Applied Mechanics  (Minor: Computer Science)%\\
% \textbf{Standardized Test Scores:} GRE -- Verbal: 650/800, Quantitative: 800/800. TOEFL -- 112/120.
&
\textbf{GPA 3.89/4.00}
\end{tabularx}

\vspace{6pt}

%--- HONORS & AWARDS
\textbf{\textsc{\LARGE{Publications, Honors \& Awards}}} \\ \rule[0.8em]{\textwidth}{0.5pt} \\[-18pt]

\begin{itemize}[topsep=0pt,partopsep=0pt,leftmargin=16pt]
\item 5 top journal articles and 15 conference proceedings (150+ citations) 
%\item 2016 Institute of Computational Engineering Sciences Postdoc Fellowship (ranked 1st)
%\item 2009 AIAA student conference Region III (Midwest) Winner 
%\item 2004 -- 2007 People's Fellowship from Fudan University First Prize (top 3\%)
\end{itemize}
%%--- LEADERSHIP
%\textbf{\textsc{\LARGE{Visa Status}}} \\ \rule[0.8em]{\textwidth}{0.5pt} \\[-18pt]
%\begin{itemize}[topsep=0pt,partopsep=0pt,leftmargin=16pt]
%\item U.S. Permanent Resident
%%\item 2004 -- 2006 People's Fellowship 2nd Prize
%\end{itemize}
%

%--- ACTIVITIES
%\textbf{\textsc{\LARGE{Memberships}}} \\ \rule[0.8em]{\textwidth}{0.5pt} \\[-20pt]

%American Physical Society \sep
%American Society of Mechanical Engineers \sep
%American Institute of Aeronautics and Astronautics \sep
%Linux Users' Group 
%National Service Scheme (India), 2007--2008 \sep
%Physics Society Chennai \sep
%Web Design \sep
%Open Source Programming\sep
%Master Swimming Club

\end{document}


% \item R.~Vishnampet, J.~B.~Freund \& D.~Bodony. Navier-Stokes adjoint accuracy for aeroacoustic flow control and analysis. {\em APS
%     Meeting Abstracts}, \textbf{57}, 2012.
% \item R.~Vishnampet \& D.~Saintillan. Concentration instability of sedimenting spheres in a second-order fluid. {\em Phys. Fluids}, \textbf{24}, 2012.
% \item A.~Narasimhan \& R.~Vishnampet. Effect of choroidal blood flow on transscleral retinal drug delivery using a porous medium
%   model. {\em Int. J. Heat Mass Transfer}, \textbf{55}:5665--5672, 2012.
% \item R.~Vishnampet, A.~Narasimhan, \& V.~Babu. High Rayleigh Number Natural Convection Inside 2D Porous Enclosures Using the
%   Lattice Boltzmann Method. {\em J. Heat Transfer}, \textbf{133}:062501, 2011.
% \item V.~Deepesh, R.~J.~Pardikar, A.~Sricharan, V.~G.~Ramanathan, S.~Chakravarthy, \& K.~Balasubramaniam. Automatic Defect Recognition System for Real Time
% Radioscopy of Hancock Valve Welds. {\em Journal of Nondestructive Testing \& Evaluation}, \textbf{9}:1--6, 2010.

% \end{enumerate}
